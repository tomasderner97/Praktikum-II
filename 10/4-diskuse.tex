\documentclass[0-protokol.tex]{subfiles}
\begin{document}
V souladu s teoretickou předpovědí jsme v obou měřeních dostali lineární závislost, žádná z hodnot se výrazně z této závislosti nevychyluje. Nejméně přesná je závislost Hallova napětí na indukci pro proud rovný $\SI{1}{A}$. Tento fakt se následně projevuje vyšší chybou vypočtené hodnoty Hallovy konstanty. Tato hodnota se v rámci chyby shoduje s hodnotou, dosaženou s využitím směrnice závislosti pro proud $\SI{4,5}{A}$.

V celém průběhu měření byly zanedbány odpory vodičů a spojů, které se však mohly v měření mírně projevit. Stejně tak není uvažován vliv prostředí. Vztah mezi napájecím proudem elektromagnetu a intenzitou magnetického pole je považován za dokonale přesný, chyba zaokrouhlení hodnoty elementárního náboje není třeba uvažovat.
\end{document}
