\documentclass[0-protokol.tex]{subfiles}
\begin{document}
V úkolu 1 byla zjištěna lineární závislost proudu na přiloženém napětí při nulové magnetické indukci.

V úkolu 2 byla taktéž naměřena lineární závislost Hallova napětí na magnetické indukci. Hodnota konstantního proudu vzorkem se odráží ve strmosti řečené závislosti.

Pomocí lineárních regresí těchto závislostí byly v úkolu 3 určeny hodnoty měrné vodivosti a Hallovy konstanty vzorku jako
$$\sigma = \SI{5,30 \pm 0,04}{\siemens\per\metre},$$
$$R_{H_{I = \SI{1}{A}}} = \SI{64,7 \pm 2,7 e-3}{\cubic\metre\per\ampere\per\second},$$
$$R_{H_{I = \SI{4,5}{A}}} = \SI{60,9 \pm 0,9 e-3}{\cubic\metre\per\ampere\per\second}.$$

V úkolu 4 následně byly spočteny hodnoty pohyblivosti a koncentrace nositelů náboje
$$n = \SI{1.171 \pm 0.012 e+20}{\per\cubic\metre},$$
$$\mu_n = \SI{0.2824 \pm 0.0022}{\cubic\metre\per\ohm\per\coulomb}.$$
\end{document}
