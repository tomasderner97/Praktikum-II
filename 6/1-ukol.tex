\documentclass[0-protokol.tex]{subfiles}
\begin{document}

\begin{enumerate}
\item Změřte účiník:
\begin{enumerate}
\item rezistoru,
\item kondenzátoru ($C = \SI{10}{\micro\farad}$),
\item cívky.
\end{enumerate}
\item Spočtěte fázový posun proudu a napětí. Určete chybu měření. Diskutujte shodu výsledků s teoretickými hodnotami pro ideální prvky.
\item Pro cívku vypočtěte indukčnost a odpor v sériovém a paralelním náhradním zapojení.
\item Změřte účiník sériového a paralelního zapojení rezistoru a kondenzátoru pro kapacity v intervalu $C = 1$ - $\SI{10}{\micro\farad}$ a spočtěte fázový posuv. Výsledky zpracujte graficky. Z naměřených hodnot stanovte odpor rezistoru a porovnejte ho s hodnotou přímo naměřenou digitálním multimetrem. Určete chyby měření a rozhodněte, které z obou zapojení je v daném případě vhodnější pro stanovení odporu.
\item Změřte závislost proudu a výkonu na velikosti kapacity zařazené do sériového RLC obvodu pro kapacity do $\SI{10}{\micro\farad}$. Výsledky zpracujte graficky, v závislosti na zařazené kapacitě vyneste účiník, fázový posuv napětí vůči proudu a výkon.
\item V průběhu měření seriového RC obvodu připojte na kondenzátor digitální osciloskop \textbf{Tektronix} a pozorujte změnu fáze napětí na kondenzátoru vzhledem k průběhu napětí zdroje v závislosti na velikosti nastavené kapacity v intervalu $1$ – $\SI{10}{\micro\farad}$. Popište kvalitativně pozorované jevy a vysvětlete je. Stručný popis ovládání a schema připojení osciloskopu je přiloženo u úlohy.
\end{enumerate}

\end{document}
