\documentclass[0-protokol.tex]{subfiles}
\begin{document}
Výsledky měření přímou metodou jsou zapsány v tabulkách \ref{tab:u1_a} a \ref{tab:u1_b} a grafech \ref{fig:u1_a}, \ref{fig:u1_a_kor}, \ref{fig:u1_b} a \ref{fig:u1_b_kor}.

Vnitřní odpory měřicích přístrojů měřené digitálním multimetrem jsou uvedeny v tabulce \ref{tab:ampermetr} a \ref{tab:voltmetr}. Hodnoty, naměřené substituční metodou, jsou pro ampérmetr s rozsahem $\SI{75}{mA}$ $\SI{4,1 \pm 0,004}{\ohm}$, pro voltmetr s rozsahem $\SI{30}{V}$ $\SI{15 \pm 0,02}{\kilo\ohm}$.

Hodnoty odporu žárovky měřené substituční metodou jsou uvedeny v tabulce \ref{tab:u3} a grafu \ref{fig:u3}.
Multimetrem jsme naměřili hodnotu $R = \SI{43.0 \pm 0.3}{\ohm}.$

Pomocí lineární regrese byla určena hodnota odporu při pokojové teplotě $R = \SI{42.35 \pm 0.09}{\ohm}.$

%Lineární regresí, zobrazenou v grafu \ref{fig:pokoj_odpor} jsem určil odpor žárovky při pokojové teplotě 
\end{document}
