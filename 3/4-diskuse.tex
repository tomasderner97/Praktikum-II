\documentclass[0-protokol.tex]{subfiles}
\begin{document}
Naměřená vstupní i výstupní charakteristika tranzistoru v zapojení se společným emitorem má charakter předpovězený literaturou, díky velkému množství naměřených hodnot se je navíc podařilo zachytit v poměrně velkém detailu. Jak je z grafu \ref{fig:u1} poznat, vstupní charakteristika tranzistoru při použití odporu $R_2 = \SI{1000}{\ohm}$ se takřka neliší od charakteristiky při rozpojení kolektorového obvodu. Naopak výstupní charakteristika je vysoce citlivá na změny proudu tekoucí bází.

Naměřená závislost zesíleného proudu $I_{CE}$ na $I_{BE}$ má podle teoretického předpokladu lineární charakter s minimální odchylkou. Taktéž v souladu s teorií je skutečnost, že pro různé hodnoty napětí $U_{CE}$ se směrnice od sebe jen málo liší. Všechny tři použité hodnoty napětí $U_{CE}$ se totiž zřejmě nacházejí v druhé, téměř nerostoucí části výstupní charakteristiky tranzistoru, tudíž pro každou hodnotu $I_{BE}$ proud kolektorem $I_{CE}$ jen málo závisí na $U_{CE}$.

Lineární závislosti $I_{CE}$ na $I_{BE}$ si lze povšimnout i z grafu \ref{fig:u2}. V celém průběhu charkteristiky jsou hodnoty $I_{CE}$ při $I_{BE} = \SI{0.1}{mA}$ zhruba dvakrát resp. třikrát menší než při $I_{BE} = \SI{0.2}{mA}$ a $\SI{0.3}{mA}$. 

Při vyhodnocení měření nebyly v úvahu brány drobné systematické chyby způsobené odpory vodičů či drobnými změnami okolních podmínek, jejich hodnoty se totiž nemohou vedle chyb měřicích přístrojů znatelně projevit.
\end{document}
