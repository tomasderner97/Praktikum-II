\documentclass[0-protokol.tex]{subfiles}
\begin{document}
Pomocí osciloskopu jsem odečetl špičkovou hodnotu napětí $U_{0, osc} = \SI{10,8 \pm 0,2}{V}$ a s pomocí multimetru jsem vypočítal špičkové napětí $U_{0, mm} = \SI{10,85 \pm 0,08}{\volt}$. Hodnoty souhlasí v rámci chyby.

Výsledky úkolu 2 lze najít v tabulkách \ref{tab:u2a} a \ref{tab:u2b} a grafech \ref{fig:u2a} a \ref{fig:u2b}.

Pomocí oscilátoru jsem sledoval V-A charakteristiku vakuové a Zenerovy diody, nákresy lze najít v přiloženém obrázku 9, respektive 10. Pro vakuovou diodu byly určeny speciální hodnoty
$$I_{U=\SI{0}{V}} = \SI{1 \pm 0,05}{mA},$$
$$U_{A=\SI{20}{mA}} = \SI{5,2 \pm 0,2}{V},$$
pro Zenerovu diodu hodnoty
$$U_{A=\SI{20}{mA}} = \SI{0,7 \pm 0,03}{V},$$
$$U_{Zener} = \SI{-6,6 \pm 0,03}{V}.$$
\end{document}
