\documentclass[0-protokol.tex]{subfiles}
\begin{document}

\begin{enumerate}
\item Pomocí osciloskopu změřte špičkovou hodnotu napětí na svorkách sekundárního vinutí transformátoru a porovnejte ji s hodnotou naměřenou na střídavém rozsahu digitálního voltmetru.
\item Podle vlastní volby sledujte činnost jednocestného nebo dvoucestného usměrňovače s křemíkovými diodami \textbf{KY711}
\begin{enumerate}
\item při maximální hodnotě zatěžovacího odporu $\SI{10}{\kilo\ohm}$ sledujte závislost stejnosměrného napětí na filtrační kapacitě $C$ v intervalu $0–10 \ \si{\micro\farad}$. Hodnotu usměrněného napětí při $C = \SI{10}{\micro\farad}$ srovnejte se špičkovou hodnotou pulzního průběhu 
\item změřte závislost filtrační kapacity $C$, potřebné k tomu, aby střídavá složka usměrněného napětí tvořila $\SI{10}{\percent}$ špičkové hodnoty (tj. asi $\SI{1}{\volt}$), na odebíraném proudu. U jednocestného usměrňovače měřte do proudu $\SI{0,6}{\milli\ampere}$, u dvoucestného do proudu $\SI{1}{\milli\ampere}$
\item naměřené závislosti zpracujte graficky. Do grafu uvádějícího závislost filtrační kapacity $C$ na proudu vyneste také závislost časové konstanty $\tau = R_zC$ na proudu.
\end{enumerate}

\end{enumerate}

\end{document}
