\documentclass[0-protokol.tex]{subfiles}
\begin{document}
Ačkoliv je osciloskop výborným pomocníkem při zobrazování průběhu okamžitého napětí, zřejmě není určen pro odečítání hodnot s vysokou přesností, jak ilustruje úloha 1, kde bylo špičkové napětí možné získat s pomocí digitálního multimetru s výrazně menší chybou. Kromě nepříliš podrobné stupnice osciloskopu ztěžoval také fakt, že použitý transformátor mírně deformoval harmonický průběh napětí.

Srovnáme-li hodnotu usměrněného napětí za použití filtrační kapacity $\SI{10}{\micro\farad}$ s hodnotou špičkového napětí měřeného v 1. úloze, vidíme rozdíl okolo 1 voltu. Tato ztráta je mimo jiné způsobena nedostatečným vyhlazením pomocí kondenzátoru, jednocestný usměrňovač není tak efektivní jako usměrňovač dvoucestný a pro shodné vyhlazení je pro jednocestný usměrňovač potřeba dvojnásobná kapacita oproti dvoucestnému.

Mírně problematické bylo měření napětí a proudu digitálními multimetry při úkolu 2, kdy se hodnoty na displeji chaoticky měnily vždy v určitém rozsahu, což reflektují zvolené hodnoty chyb. Toto chování je do jisté míry vysvětleno nízkou kvalitou použitého transformátoru a použitím jednocestného usměrňovače.

Přesto, že odečítání významných hodnot v úkolu 3 bylo ztíženo okolnostmi, popsanými ve výsledcích měření, použitím různých rozsahů a poloh počátku souřadnic, jakožto i nastavením optimálního napětí na potenciometru bylo dosaženo uspokojivé přesnosti.

Chyby, způsobené podmínkami okolí, odpory vodičů či změny odporů způsobené zahříváním, nejsou při tomto měření vzhledem k ostatním chybám podstatné, jejich důsledky tedy neuvažuji.
\end{document}
