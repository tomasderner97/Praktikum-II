\documentclass[0-protokol.tex]{subfiles}
\begin{document}
Z grafů je patrné, že zatímco molární vodivost kyseliny chlorovodíkové je téměř konstantní (čímž se dá také vysvětlit na první pohled nelineární vzezření grafu \ref{fig:silny}, směrnice přímky je natolik malá, že i drobné odchylky od lineárního průběhu se mohou graficky výrazně projevit), molární vodivost kyseliny octové značně klesá s rostoucí koncentrací roztoku. Více se tam tedy projevují vzájemné interakce mezi ionty.

Při tomto měření bylo zanedbáno značné množství zdrojů systematických chyb, které se však mohly reálně projevit. Zmiňme především proměnlivost teploty okolí v průběhu experimentu až o $\SI{1}{\celsius}$. K tomuto přistupuje skutečnost, že nebyla ověřena správnost kalibrace konduktometru. V neposlední řadě lze zmínit, že určení chyby měření měrné vodivosti pomocí metody popsané ve výsledcích měření není z principu příliš přesné.
\end{document}
