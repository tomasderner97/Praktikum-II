\documentclass[0-protokol.tex]{subfiles}
\begin{document}
Při tomto měření je velké monžství možných chyb spojeno s nedokonalou teplotní rovnováhou v měřicí soustavě, především pak při měření teplotní závislosti odporu. Teploměr totiž nemohl být na přesně stejném místě jako termistor, jistě tedy vzniká určitá systematická chyba v určení teploty, která se v uvedených hodnotách chyb neodráží. V průběhu měření statické charakteristiky pak mohlo vlivem malé prodlevy mezi změnou protékajícího proudu a odečtením měřených hodnot dojít k určitému nadhodnocení měřeného napětí.

K výše uvedeným příčinám chyb také přistupuje skutečnost, že teplota okolí nebyla měřena přímo, ale odečtena později z lineární regrese lineární části statické charakteristiky, která mohla být již zmíněným způsobem zkreslená.
\end{document}
