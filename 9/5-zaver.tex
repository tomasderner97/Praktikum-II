\documentclass[0-protokol.tex]{subfiles}
\begin{document}
Křivka naměřené statické charakteristiky termistoru v první části lineárně roste, následně mírně klesá. Naměřená teplotní závislost má exponenciální průběh, obě závislosti odpovídají teoretické předpovědi.

Z lineární regrese podle rovnice \eqref{eq:log_R} jsme určili konstanty
$$ B = \SI{2610 \pm 12}{\kelvin}, $$
$$ R_\infty = \SI{0.086 \pm 0.004}{\ohm}, $$
dále byla spočtena aktivační energie 
$$ \Delta U = \SI{43.40 \pm 0.19 e3}{\joule\per\mole} $$
a teplotní součinitel odporu termistoru při pokojové teplotě
$$ \alpha = \SI{29.4 \pm 0.2 e-3}{\per\kelvin}. $$

Ze statické charakteristiky byly spočteny hodnoty teploty termistoru v maximu charakteristiky a tepelný odpor
$$ T_m = \SI{339.7 \pm 0.7}{\kelvin}, $$
$$ K = \SI{7.14 \pm 0.08 e3}{\kelvin\per\watt}. $$

\end{document}
